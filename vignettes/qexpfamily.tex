% \VignetteIndexEntry{Q-exponential family}
% \VignetteKeywords{htest}
% \VignettePackage{tsallisqexp}

%----------------------------------------------------------------------------------------------------------
% 	Copyright (c) 2011 Christophe Dutang, 
% 	
% 	The following Latex code is given as an example and not The example
%	of using Latex to write report, memoir or thesis. 
%
%	The licence is unlimited but the copyright authors could NOT be liable
%	for any direct or indirect damages caused by the use of the following code.
%
%	The following code could be compiled :
%	  - on windows with a Tex distribution such as miktex (http://miktex.org) 
%		and a front end Latex editor such as texniccenter (http://www.toolscenter.org)
%	  - on mac os with a Tex distribution such as TexLive and a front end Latex
%	  	editor such as Texshop (http://www.uoregon.edu/~koch/texshop/)
%	  - on linux with a Tex distribution such as teTex (http://www.tug.org/teTeX/)
%	  	and a front end Latex editor such as emacs (http://www.gnu.org/software/emacs/)
%
%	  We may use \'e (\`e, \`a,...) instead of the accented character to ensure the 
%	    portability of the Latex code. This is NOT an obligation of Latex.
%
%----------------------------------------------------------------------------------------------------------

\documentclass[11pt]{article}

% package
%symbole math de l'American Mathematical Society (AMS)
\usepackage{amsfonts,amssymb,amsmath,amsthm}
%utiliser les regles de typographie francaise
\usepackage[english]{babel}
\usepackage{courier,newlfont}

% accents 8 bits dans le fichier
%\usepackage[applemac]{inputenc} %MAC encoding
%\usepackage[utf8]{inputenc} %UNIX encoding
\usepackage[utf8x]{inputenc} %UNIX encoding on MAC
%\usepackage[ansinew]{inputenc} %WINDOWS encoding

%graphique
\usepackage[dvips]{graphicx}
\usepackage{color,graphics, wrapfig, subfig}

\usepackage{tabls}

%citation
\usepackage{natbib}

%short toc
\usepackage{shorttoc}

%reference hypertext
\usepackage[pagebackref=true, hyperfootnotes=false]{hyperref}

%note de bas de page
\usepackage[symbol,stable]{footmisc}
\usepackage{perpage}
\MakePerPage[1]{footnote}

%header
%\pagestyle{headings}

%pour les url
\usepackage{url}
\urlstyle{sf}

%plusieur colonne dans tableau
\usepackage{multirow, multicol}

% todo's
\usepackage{todonotes}
% usage: \todo[color=green]{TODO}blabla 



% layout sinon utiliser vmargin
\normalsize\setlength{\parskip}{\baselineskip}
\setlength{\oddsidemargin}{20mm}
\setlength{\evensidemargin}{20mm}
\setlength{\voffset}{-1in}
\setlength{\hoffset}{-1in}
\setlength{\textwidth}{175mm}
\setlength{\topmargin}{0mm}
\setlength{\headheight}{05mm}
\setlength{\headsep}{05mm}
\setlength{\topskip}{0mm}
\setlength{\textheight}{250mm}




% les macros generales

%layout
\newcommand{\HRuleTop}{\noindent\rule{\linewidth}{.5pt}}
\newcommand{\HRuleBottom}{\rule{\linewidth}{.5pt}}
\newcommand{\ligne}{\rule[2mm]{.3\textwidth}{0,5mm}\\}
\newcommand{\myskip}{\vspace{\parskip}}
\newcommand{\mytodo}[1]{\todo[color=green]{TODO}#1}
\newcommand{\blank}{ \clearpage{\pagestyle{empty}\cleardoublepage} }


%text style
\newcommand{\pkg}{\textbf}
\newcommand{\sigle}{\textsc}
\newcommand{\code}{\texttt}
\newcommand{\soft}{\textsf}
\newcommand{\txtm}[1]{\textrm{~~#1~~}}
\newcommand{\expo}{\textsuperscript}

%\newcommand{\(}{\left(}


%system
\newcommand{\systL}{\left\{\begin{array}{l}}
\newcommand{\systR}{\end{array}\right.}
\newcommand{\matL}{\left(\begin{matrix}}
\newcommand{\matR}{\end{matrix}\right)}
\newcommand{\detL}{\left|\begin{matrix}}
\newcommand{\detR}{\end{matrix}\right|}



%sets
\newcommand{\R}{\mathbb R}
\newcommand{\N}{\mathbb N}
\newcommand{\Z}{\mathbb Z}
\newcommand{\X}{\mathbb X}
\newcommand{\C}{\mathbb C}
\newcommand{\K}{\mathbb K}
\newcommand{\Rnm}{\R^{n\times m}}
\newcommand{\Cnm}{\C^{n\times m}}
\newcommand{\Knm}{\K^{n\times m}}
\newcommand{\Kmn}{\K^{m\times n}}
\newcommand{\Knn}{\K^{n\times n}}
\newcommand{\Kmp}{\K^{m\times p}}




\title{On the different parametrizations of the Q-exponential family distribution}
\author{C. Dutang\footnote{LMM, Universit\'e du Maine, F-Le Mans city}, P. Higbie\footnote{NMSU, Mexico}}



\usepackage{Sweave}
\begin{document}
\Sconcordance{concordance:qexpfamily.tex:qexpfamily.Rnw:%
1 147 1 1 0 137 1}


\maketitle

\section{q-exponential family}
The density (Eq. (18) of Naudt (2007)) is defined as
$$
f_\theta(x) = c(x) \exp_q(-\alpha(\theta)-\theta H(x)),
$$
where $c$, $\alpha$ and $H$ are known functions.
Furthermore, $\exp_q$ is the $q$-deformed exponential function defined as
$$
\exp_q(z)= [1+(1-q)z]_+^{1/(1-q)} \text{ for } z\in\R, q\neq 1,
$$
where $[z]_+=\max(z,0)$. 
$\exp_q$ is construct as the inverse of the $q$-deformed logarithm defined as
$$
\log_q(z) = \frac{z^{1-q}-1}{1-q} \text{ for } z\in\R, q\neq 1.
$$
In particular, $\forall z\in\R, \exp_q(\log_q(z)) =z$ and $\forall z\neq0, \log_q(\exp_q(z)) =z$.
Special case: for $q\rightarrow1$, $\exp_q\rightarrow \exp$ and we get the exponential family.


Let us find the domain where $1+(1-q)z>0$:
\begin{itemize}
\item
If $q>1$, i.e. $1-q< 0$ then
$$
1+(1-q) z >0
\Leftrightarrow 1>-(1-q) z
\Leftrightarrow \frac{-1}{1-q}>z
$$
\item
If $q<1$, i.e. $1-q> 0$ then
$$
1+(1-q) z >0
\Leftrightarrow 1>-(1-q) z
\Leftrightarrow \frac{1}{1-q}<z
$$
\end{itemize}




\newpage
\section{q-Gaussian}
Using 
$$
c(x) = 1/c_q, ~
c_q = \sqrt{\frac{\pi}{1-q}} \frac{\Gamma(1+1/(1-q))}{\Gamma(3/2+1/(1-q))},~
H(x)=x^2,~
\alpha(\theta) = \frac{\theta^{\frac{q-1}{3-q}}-1}{q-1},~
\theta=\sigma^{q-3},
$$
we have $\alpha(\sigma) = \frac{\sigma^{q-1}-1}{q-1}=\log_{2-q}(\sigma)$.
We get
\begin{eqnarray*}
f_\sigma(x) 
&=& \frac{1}{c_q} \exp_q(-\log_{2-q}(\sigma) - x^2\sigma^{q-3})
= \frac{1}{c_q} \exp_q(- \frac{\sigma^{-1+q}-1}{-1+q} - x^2\sigma^{q-3})
\\
&=& \frac{1}{c_q} \exp_q( \frac{(1/\sigma)^{1-q}-1}{1-q} - x^2\sigma^{q-3})
= \frac{1}{c_q} \left[1+(1-q)\frac{(1/\sigma)^{1-q}-1}{1-q} - (1-q)x^2\sigma^{q-3}\right]_+^{1/(1-q)}
\\
&=&
\frac{1}{c_q} \left[(1/\sigma)^{1-q} - (1-q)x^2\sigma^{q-3}\right]_+^{1/(1-q)}
=\frac{1}{c_q\sigma} \left[ 1 -  (1-q)\sigma^{1-q} x^2\sigma^{q-3}\right]_+^{1/(1-q)}  
\\
&=&
\frac{1}{c_q\sigma} \left[ 1 -  (1-q)x^2/\sigma^{2} \right]_+^{1/(1-q)}  
= \frac{1}{c_q\sigma}\exp_q(-x^2/\sigma^2)
\end{eqnarray*}
This is different from Section 6 of Naudt (2007) where there is a typo.





\section{q-Exponential}
Using 
$$
c(x) = 1/c_q, ~
c_q = \sqrt \kappa,~
H(x)=x,~
\alpha(\theta) = \frac{\theta^{\frac{q-1}{3-q}}-1}{q-1},~
\theta=\kappa^{\frac{q-3}{2}},
$$
we get
$$
f_\kappa(x) 
= \frac{1}{ \kappa}\exp_q(-x/\kappa)
= \frac{1}{ \kappa}\left(1-(1-q) \frac{x}{\kappa}\right)_+^{1/(1-q)}.
$$
There exists another parametrization 
$$
\systL
\alpha+1 = -1/(1-q) \\ %1/(alpha+1) = -(1-q) => 1-1/(alpha+1)=q
\sigma = \alpha\kappa 
\systR
\Leftrightarrow
\systL
\sigma = \alpha\kappa \\
-1/(\alpha+1) = 1-q \\ 
\systR
\Leftrightarrow
\systL
\kappa = \sigma/\alpha \\
q = 1 + 1/(\alpha+1)
\systR
$$
Using the parametrization $(\alpha,\sigma)$, we get the following density and distribution function
$$
f(x) = \frac{\alpha}{ \sigma}\left(1+ \frac{x}{\sigma}\right)_+^{-\alpha-1},~
F(x) = 1- \left(1+ \frac{x}{\sigma}\right)_+^{-\alpha} .
$$




%%%%%%%%%%%%%%%%%%%%%%%%%%%%%%%%%%%%%%%%%%%%
%  Bibiliography.
%%%%%%%%%%%%%%%%%%%%%%%%%%%%%%%%%%%%%%%%%%%%
\section{Bibliography}

\noindent
Naudt, J. (2007),
{\it The q-exponential family in statistical physics}, 
Journal of Physics: Conference Series 201 (2010) 012003.

\noindent
Shalizi, C. (2007),
{\it Maximum Likelihood Estimation for q-Exponential (Tsallis) Distributions}, 
\url{http://arxiv.org/abs/math/0701854v2}.



\end{document}
